\documentclass[a4paper,10pt]{report}

\usepackage{packages/rapportutc}
%\usepackage{packages/include-packages}


%%%%%%%% Références %%%%%%%%
\usepackage{cleveref}
\usepackage{hyperref}
\usepackage[nottoc, notlof, notlot]{tocbibind} % bibliographie http://tex.stackexchange.com/questions/71129/bibliography-in-table-of-contents
\usepackage{natbib} % bibliographie

%%%%%%%% Code informatique %%%%%%%%
\usepackage{packages/Sweave} %package d'affichage des codes R
\usepackage{listings} % pour hightlight code

%%%%%%%% Formules mathématiques %%%%%%%%
\usepackage{amsmath, amsthm, amssymb, graphics, setspace} %packages de mathématiques
\usepackage{chemist} %formule chimique 

%%%%%%%% Mise en forme %%%%%%%%
% Mise en forme graphique
\usepackage{graphicx,wrapfig,lipsum} % pour afficher des figures à côté du texte
\usepackage[linewidth=1pt]{mdframed} % permet de générer et gérer des frames
\usepackage{rotating} % rotations on tables, captions, text, ...
% Mise en forme images et tableaux
\usepackage{float} % permet de spécifier l'option "H" aux captions afin de les positionner de manière fixe
\usepackage{subcaption} % permet d'afficher plusieurs images dans une caption
\usepackage{array} % meilleurs "table" et "tabular"
% Mise en forme texte
\usepackage{setspace} % permet de spécifier l'espacement interligne
\usepackage{ulem} % \sout{Texte à barrer} \xout{Texte à hachurer} \uwave{Texte à souligner par une vaguelette}
\usepackage{calc,enumitem}  % Mise en forme l'environnement itemsize description etc.
\usepackage{color} % utilisation de couleurs
\usepackage{ae,aecompl} % Vir­tual fonts for T1 en­coded CMR-fonts
\usepackage{pifont} % com­mands for Pi fonts (Ding­bats, Sym­bol, etc.)
\usepackage{comment} % Selectively include/exclude portions of text \comment....\endcomment

\onehalfspacing % espacement interligne
\setlength{\parindent}{.5em} % indentation des retraits de première ligne

%%%%%%%%%%%%%%%%%%%%%%%%%%%%%%%%%%%%%%%%%%%%%%%%%%%%%%%%%%%%%%%%%%%%%%%%%%%% 

\title{TP 1 - évolution de la valeur d'un actif financier}
\author{LU Han - SAUVENT Alexandre}
\date{\today}

\uv{RO05}
\branche{Génie Informatique}
\filiere{Fouille de Données et Décisionnel}
%%%%%%%%%%%%%%%%%%%%%%%%%%%%%%%%%%%%%%%%%%%%%%%%%%%%%%%%%%%%%%%%%%%%%%%%%%%%

\begin{document}

\renewcommand{\labelitemi}{\large\textcolor{tatoebagreen}{\fg}}
\newgeometry{top=2.5cm,bottom=2cm,left=2cm,right=2cm}
\groovypdtitre
\restoregeometry % restaure la géométrie par défaut de latex

%%%%%%%%%%%%%%%%%%%%%%%%%%%%%%%%%%%%%%%%%%%%%%%%%%%%%%%%%%%%%%%%%%%%%%%%%%%% 

\tableofcontents

%%%%%%%%%%%%%%%%%%%%%%%%%%%%%%%%%%%%%%%%%%%%%%%%%%%%%%%%%%%%%%%%%%%%%%%%%%%%

\chapter{Contexte}
\noindent Considérons une option Européenne représentant un actif financier dont la valeur au temps \emph{t} est S(t) pour $0 \leq t \leq T$, où T est le temps de l'exercice de l'option. Notons que $M \epsilon N^{\ast}$ le nombre de subdivision de l'intervalle [0,T] et h = T/M. notons également $S_{n} = S(nh)$, pour n = 0,1,...,M, les valeurs de l'actif aux instants t = nh.
L‘équation d'évolution du prix de l'actif en temps discret s'écrit comme suit:
\begin{equation}
S_{n+1} = S_{n} + \mu h S_{n} + \sigma h^{1/2} S_{n} \xi_{n}   
\end{equation} 
où $\xi_{n}, n \geq 0$ est une suite de v.a. iid de loi N(0,1) indépendante de $S_{0}$. 

\chapter{Démonstration des exercices}
\section{Question 1}
\noindent Monter que 
\begin{align*}
	S_{M} = S_{0} \prod_{n=0}^{M-1} (1 + \mu h + \sigma h^{1/2} \xi_{n}),
\end{align*}
\noindent et en réduire
\begin{align*}
	ln(\frac{S_{M}}{S_{0}}) = \sum_{n=0}^{M-1} ln(1  + \mu h + \sigma h^{1/2} \xi_{n}).
\end{align*}
\noindent Démonstration:
\par Selon l'équation(1.1), nous utilisons la méthode de la multiplication continuée pour résoudre cette question:
\par quand n = 0, nous obtenons:
\begin{equation}
	S_{1} = S_{0} + \mu h S_{0} + \sigma h^{1/2} S_{0} \xi_{0}
\end{equation} 
\par et pour $S_{0} \neq 0$ nous obtenons:
\begin{equation}
\frac{S_{1}}{S_{0}} = 1 + \mu h + \sigma h^{1/2} \xi_{0}
\end{equation} 
\par quand n = 1, nous obtenons:
\begin{equation}
S_{2} = S_{1} + \mu h S_{1} + \sigma h^{1/2} S_{1} \xi_{1}
\end{equation}
\par et pour $S_{1} \neq 0$ nous obtenons:
\begin{equation}
\frac{S_{2}}{S_{1}} = 1 + \mu h + \sigma h^{1/2} \xi_{1}
\end{equation} 	

\end{document}
